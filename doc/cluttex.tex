\documentclass[a4paper]{report}
\usepackage{verbatim}
\usepackage[unicode]{hyperref}
\usepackage{amsmath}
\newcommand\ClutTeX{Clut\TeX}
\providecommand\BibTeX{\textsc{Bib}\TeX}
\newcommand\texcmd[1]{\texttt{\textbackslash #1}}
\newcommand\texenv[1]{\texttt{#1}}
\newcommand\texpkg[1]{\texttt{#1}}
\newcommand\metavar[1]{\textnormal{\textsf{#1}}}

\title{\ClutTeX\ manual\\(Version 0.7.0)}
\author{ARATA Mizuki}
\date{2025-03-16}

\begin{document}
\maketitle
\tableofcontents

\chapter{About \ClutTeX}
\ClutTeX\ is an automation tool for \LaTeX\ document processing.
Basic features are,
\begin{itemize}
\item Does not clutter your working directory with ``extra'' files, like \texttt{.aux} or \texttt{.log}.
\item If multiple runs are required to generate correct document, do so.
\item Watch input files, and re-process documents if changes are detected\footnote{needs an external program if you are on a Unix system}.
\item Run MakeIndex, \BibTeX, Biber, if requested.
\item Produces a PDF, even if the engine (e.g.\ p\TeX) does not suport direct PDF generation.
  If you want a DVI file, use \texttt{--output-format=dvi} option.
\end{itemize}

The unique feature of this program is that, auxiliary files such as \texttt{.aux} or \texttt{.toc} are created in an isolated location, so you will not be annoyed with these extra files.

% A competitor: \href{http://www.personal.psu.edu/jcc8/latexmk/}{Latexmk}

\chapter{How to use \ClutTeX}
\section{Installation}
If you are using the latest \TeX\ Live, you should have \ClutTeX\ installed.
If not, upgrade your copy of \TeX\ Live with \texttt{tlmgr update --all}.

If you want to install \ClutTeX\ manually, fetch an archive from GitHub\footnote{\url{https://github.com/minoki/cluttex}}, extract it, and copy \texttt{bin/cluttex} or \texttt{bin/cluttex.bat} to somewhere in your \texttt{PATH}.

\section{Command-line usage}
Usage:
\begin{center}
  \texttt{cluttex -e \metavar{ENGINE} \metavar{OPTIONs} [--] \metavar{INPUT}.tex}
\end{center}

Basic options:
\begin{description}
\item[\texttt{-e}, \texttt{--engine=\metavar{ENGINE}}]
  Set which \TeX\ engine/format to use.
  \metavar{ENGINE} is one of the following:
  \texttt{pdflatex}, \texttt{pdftex},
  \texttt{lualatex}, \texttt{luatex}, \texttt{luajittex},
  \texttt{xelatex}, \texttt{xetex},
  \texttt{latex}, \texttt{etex}, \texttt{tex},
  \texttt{platex}, \texttt{eptex}, \texttt{ptex},
  \texttt{uplatex}, \texttt{euptex}, or \texttt{uptex}.
  Required.
\item[\texttt{-o}, \texttt{--output=\metavar{FILE}}]
  Set output file name.
  Default: \texttt{\metavar{JOBNAME}.\metavar{FORMAT}}
\item[\texttt{--fresh}]
  Clean auxiliary files before run.
  Cannot be used in conjunction with \texttt{--output-directory}.
\item[\texttt{--max-iterations=\metavar{N}}]
  Set maximum number of run, for resolving cross-references and etc.
  Default: 4
\item[\texttt{--watch[=\metavar{ENGINE}]}]
  Watch input files for change.
  May need an external program to be available.
  See \autoref{sec:watch-mode} for details.
\item[\texttt{--color[=\metavar{WHEN}]}]
  Colorize messages.
  \metavar{WHEN} is one of \texttt{always}, \texttt{auto}, or \texttt{never}.
  If \texttt{--color} option is omitted, \texttt{auto} is used.
  If \metavar{WHEN} is omitted, \texttt{always} is used.
\item[\texttt{--includeonly=\metavar{NAMEs}}]
  Insert \texttt{\texcmd{includeonly}\{\metavar{NAMEs}\}}.
\item[\texttt{--make-depends=\metavar{FILE}}]
  Write Makefile-style dependencies information to \metavar{FILE}.
\item[\texttt{--engine-executable=\metavar{COMMAND}}]
  The actual \TeX\ command to use.
\item[\texttt{--tex-option=\metavar{OPTION}}, \texttt{--tex-options=\metavar{OPTIONs}}]
  Pass extra options to \TeX.
\item[\texttt{--dvipdfmx-option=\metavar{OPTION}}, \texttt{--dvipdfmx-options=\metavar{OPTIONs}}]
  Pass extra options to \texttt{dvipdfmx}.
\item[\texttt{--[no-]change-directory}]
  Change to the output directory when run.
  May be useful with shell-escaping packages.
\item[\texttt{-h}, \texttt{--help}]
\item[\texttt{-v}, \texttt{--version}]
\item[\texttt{-V}, \texttt{--verbose}]
\item[\texttt{--print-output-directory}]
  Print the output directory and exit.
\item[\texttt{--package-support=PKG1[,PKG2,...,PKGn]}]
  Enable special support for shell-escaping packages.
  Currently, supported packages are `\texttt{minted}', `\texttt{epstopdf}' and `\texttt{pdfx}'.
\item[\texttt{--check-driver=DRIVER}]
  Check that the correct driver file is loaded for certain packages.
  \metavar{DRIVER} is one of \texttt{dvipdfmx}, \texttt{dvips}, or \texttt{dvisvgm}.
  Can only be used with \texttt{--output-format=dvi}.
\item[\texttt{--source-date-epoch=TIME}]
  Lock the creation time of PDF by setting the environment variable \texttt{SOURCE\_DATE\_EPOCH}.
  \metavar{TIME} is `\texttt{now}' or an unsigned integer (typically, this value is interpreted as a Unix time).
\item[\texttt{--config-file=FILE}]
  Use the specified config file. See \autoref{sec:config-file}.
\end{description}

Options for running auxiliary programs:
\begin{description}
\item[\texttt{--makeindex=\metavar{COMMAND}}]
  Run MakeIndex.
\item[\texttt{--bibtex=\metavar{COMMAND}}]
  Run \BibTeX.
\item[\texttt{--biber[=\metavar{COMMAND}]}]
  Run Biber. Default value for \metavar{COMMAND}: \texttt{biber}
\item[\texttt{--makeglossaries[=\metavar{COMMAND}]}]
  Run makeglossaries. Experimental.
\end{description}

\TeX-compatible options:
\begin{description}
\item[\texttt{--[no-]shell-escape}]
\item[\texttt{--shell-restricted}]
\item[\texttt{--synctex=\metavar{NUMBER}}]
  Generate Sync\TeX\ file.
  Note that \texttt{.synctex.gz} is created alongside the final \texttt{.pdf}.
  See \autoref{sec:synctex} for details.
\item[\texttt{--[no-]file-line-error}]
  Default: Yes
\item[\texttt{--[no-]halt-on-error}]
  Default: Yes
\item[\texttt{--interaction=\metavar{STRING}}]
  \metavar{STRING} is one of \texttt{batchmode}, \texttt{nonstopmode}, \texttt{scrollmode}, or \texttt{errorstopmode}.
  Default: \texttt{nonstopmode}
\item[\texttt{--jobname=\metavar{STRING}}]
\item[\texttt{--fmt=\metavar{FORMAT}}]
\item[\texttt{--output-directory=\metavar{DIR}}]
  Set output directory for \TeX\ engine.
  Auxiliary files are produced in this directory.
  Default: Based on the \texttt{temporary-directory} configuration.
\item[\texttt{--output-format=\metavar{FORMAT}}]
  Set output format.
  Possible values are \texttt{pdf} or \texttt{dvi}.
  Default: \texttt{pdf}
\end{description}

Long options, except \TeX-compatible ones, need two hyphens (e.g. \texttt{-synctex=1} is accepted, but not \texttt{--color}).
Combining multiple short options, like \texttt{-Ve pdflatex}, is not supported.

\section{Sync\TeX}\label{sec:synctex}
You can generate Sync\TeX\ data with \texttt{--synctex=1} option.

Although \ClutTeX\ has ``Don't clutter your working directory'' as its motto, the \texttt{.synctex.gz} file is always produced alongside the PDF file.
This is because Sync\TeX\ cannot find its data file if it's not in the same directory as the PDF.

\section{Watch mode}\label{sec:watch-mode}
If \texttt{--watch} option is given, \ClutTeX\ enters \emph{watch mode} after processing the document.

On Windows, a built-in filesystem watcher is implemented.
On other platforms, an auxiliary program \texttt{fswatch}\footnote{\url{http://emcrisostomo.github.io/fswatch/}} or \texttt{inotifywait} needs to be installed.
The auxiliary program will be detected automatically, but it could also be specified by the \metavar{ENGINE} argument.

\section{MakeIndex and \BibTeX}
If you want to generate index or bibliography, using MakeIndex or \BibTeX, set \texttt{--makeindex}, \texttt{--bibtex}, or \texttt{--biber} option.
You need to explicitly specify the command name as an argument (e.g. \texttt{--makeindex=makeindex}, \texttt{--bibtex=bibtex}).

If you want to use Biber to process bibliography, the option to use is \texttt{--biber}, not \texttt{--bibtex=biber}.

\section{For writing a large document}
When writing a large document with \LaTeX, you usually split the \TeX\ files with \texcmd{include} command.
When doing so, \texcmd{includeonly} can be used to eliminate processing time.
But writing \texcmd{includeonly} in the \TeX\ source file is somewhat inconvenient.
After all, \texcmd{includeonly} is about \emph{how} to process the document, not about its content.

Therefore, \ClutTeX\ provides an command-line option to use \texcmd{includeonly}.
See \autoref{sec:makefile-example} for example.

Tips: When using \texttt{includeonly}, avoid using \texttt{--makeindex} or \texttt{--biber}.

Another technique for eliminating time is, setting \texttt{--max-iterations=1}.
It stops \ClutTeX\ from processing the document multiple times, which may take several extra minutes.

\section{Using Makefile}\label{sec:makefile-example}
You can create Makefile to avoid writing \ClutTeX\ options each time.
Example:
\begin{verbatim}
main.pdf: main.tex chap1.tex chap2.tex
    cluttex -e lualatex -o $@ --makeindex=mendex $<

main-preview.pdf: main.tex chap1.tex chap2.tex
    cluttex -e lualatex -o $@ --makeindex=mendex --max-iterations=1 $<

chap1-preview.pdf: main.tex chap1.tex
    cluttex -e lualatex -o $@ --max-iterations=1 --includeonly=chap1 $<

chap2-preview.pdf: main.tex chap2.tex
    cluttex -e lualatex -o $@ --max-iterations=1 --includeonly=chap2 $<
\end{verbatim}

With \texttt{--make-depends} option, you can let \ClutTeX\ infer sub-files and omit them from Makefile.
Example:

\begin{verbatim}
main.pdf: main.tex
    cluttex -e lualatex -o $@ --make-depends=main.pdf.dep $<

-include main.pdf.dep
\end{verbatim}

After initial \texttt{make} run, \texttt{main.pdf.dep} will contain something like this:
\begin{verbatim}
main.pdf: ... main.tex ... chap1.tex chap2.tex
\end{verbatim}

Note that \texttt{--make-depends} option is still experimental, and may not work well with other options like \texttt{--makeindex}.

\section{Default output directory}
The auxiliary files like \texttt{.aux} are generated somewhere in the temporary directory, by default.
The directory name depends on the following three parameters:
\begin{itemize}
\item The absolute path of the input file
\item \texttt{--jobname} option
\item \texttt{--engine} option
\end{itemize}
On the other hand, the following parameters doesn't affect the directory name:
\begin{itemize}
\item \texttt{--includeonly}
\item \texttt{--makeindex}, \texttt{--bibtex}, \texttt{--biber}, \texttt{--makeglossaries}
\end{itemize}

If you need to know the exact location of the automatically-generated output directory, you can invoke \ClutTeX\ with \texttt{--print-output-directory}.
For example, \texttt{clean} target of your Makefile could be written as:
\begin{verbatim}
clean:
    -rm -rf $(shell cluttex -e pdflatex --print-output-directory main.tex)
\end{verbatim}

\ClutTeX\ itself doesn't erase the auxiliary files, unless \texttt{--fresh} option is set.
Note that, the use of a temporary directory means, the auxiliary files may be cleared when the computer is rebooted.

\section{Aliases}
Some Unix commands change its behavior when it is called under a different name.
There are several examples in \TeX\ Live:
\begin{itemize}
\item \texttt{extractbb} and \texttt{dvipdfmx} are aliases for \texttt{xdvipdfmx}.
\item \texttt{repstopdf} is an alias for \texttt{epstopdf}.
\end{itemize}

If \ClutTeX\ is called as \texttt{cl}\(\langle\text{\metavar{ENGINE}}\rangle\), the \texttt{--engine} option is set accordingly.
For example, \texttt{cllualatex} is an alias for \texttt{cluttex --engine lualatex} and \texttt{clxelatex} for \texttt{cluttex --engine xelatex}.

% The aliases provided by \TeX\ Live are, \texttt{cllualatex} and \texttt{clxelatex}.

\section{Support for \texpkg{minted} and \texpkg{epstopdf}}
In general, packages that execute external commands (shell-escape) don't work well with \texttt{-output-directory}.
Therefore, they don't work well with \ClutTeX.

However, some packages provide a package option to let them know the location of \texttt{-output-directory}.
For example, \texpkg{minted} provides \texttt{outputdir}, and \texpkg{epstopdf} provides \texttt{outdir}.

\ClutTeX\ can supply them the appropriate options, but only if it knows that the package is going to be used.
To let \ClutTeX\ what packages are going to be used, use \texttt{--package-support} option.

For example, if you want to typeset a document that uses \texpkg{minted}, run the following:
\begin{verbatim}
cluttex -e pdflatex --shell-escape --package-support=minted document.tex
\end{verbatim}

\section{Check for driver file}

\ClutTeX\ can check that the correct driver file is loaded when certain packages are loaded.
Currently, the list of supported packages are \texpkg{graphics}, \texpkg{color}, \texpkg{expl3}, \texpkg{hyperref}, and \texpkg{xy}.

The check is always done with PDF mode.
To check the driver with DVI mode, use \texttt{--check-driver} option.

\section{Customization via a configuration file}\label{sec:config-file}

Some behavior of \ClutTeX\ can be customized via a configuration file.

The configuration file for \ClutTeX\ is looked for in the following order:
\begin{enumerate}
\item \texttt{--config-file} command-line option.
\item \texttt{CLUTTEX\_CONFIG\_FILE} environment variable.
\item (Unix only) \texttt{\$XDG\_CONFIG\_HOME/cluttex/config.toml}
\item (Unix only) \texttt{\$HOME/.config/cluttex/config.toml}
\item (Windows only) \texttt{\%APPDATA\%\textbackslash cluttex\textbackslash config.toml}
\end{enumerate}

The configuration file for \ClutTeX\ must be written in TOML\footnote{\url{https://toml.io/en/}} format.

Available keys are:
\begin{description}
\item[\texttt{temporary-directory} (string)]
  The location to create auxiliary files.
  The default value for \texttt{--output-directory} will be somewhere under \texttt{temporary-directory}.
  If omitted, somewhere in the temporary directory will be used.
\item[\texttt{color.\{type,execute,warning,diagnostic,information\}} (table)]
  Set the colors for messages from \ClutTeX.
  The table may have the following keys:
  \texttt{fore} (color string),
  \texttt{back} (color string),
  \texttt{bold} (boolean),
  \texttt{dim} (boolean),
  \texttt{underline} (boolean),
  \texttt{blink} (boolean),
  \texttt{reverse} (boolean),
  \texttt{italic} (boolean),
  \texttt{strike} (boolean).
  A color string is one of the following:
  \texttt{default},
  \texttt{black},
  \texttt{red},
  \texttt{green},
  \texttt{yellow},
  \texttt{blue},
  \texttt{magenta},
  \texttt{cyan},
  \texttt{white},
  \texttt{brightblack},
  \texttt{brightred},
  \texttt{brightgreen},
  \texttt{brightyellow},
  \texttt{brightblue},
  \texttt{brightmagenta},
  \texttt{brightcyan},
  \texttt{brightwhite}.
\end{description}

An example configuration file might look like:
\begin{verbatim}
# The actual output directory will be something like
# "/home/user/.cache/cluttex/cluttex-<hash>"
temporary-directory = "/home/user/.cache/cluttex"

[color]
type = { reverse = true }
error = { fore = "brightred", back = "brightwhite" }
\end{verbatim}

\end{document}
